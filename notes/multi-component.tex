\documentclass[11pt]{article}
\usepackage{charter}
\usepackage{graphicx}

\title{Multi-Component Model for Tactoids}
\author{Emmanuel Flores}

\begin{document}
\maketitle
Tactoid are non-spherical droplets of liquid crystals inmersed in an isotropic fluid. The standard description of these shapes is based on a free energy with elastic, surface energy and anchoring contributions, where different types of tactoids, such as bipolar or uniform are set by a compentence in the former energies. 
The following notes aim to propose a model in which we have a tactoid inside another liquid crystal.

As stated previously, in the ausence of topological defects, the standard approach to follow is a director formulation, in which the free energy takes the form

\begin{equation}
F = F_{e} + F_{st} + F_{a},
\end{equation}
and the notation I'll be using is $F = \int dx\mathcal{F}$, where each contribution is given by

\begin{equation}
F_{e} = \frac{1}{2}K_{11}(\nabla\cdot\mathbf{n})^2 + \frac{1}{2}K_{22}\left( \mathbf{n}\cdot \nabla\times \mathbf{n}\right)^2 + \frac{1}{2}K_{33}\left( \mathbf{n}\times\nabla\times\mathbf{n}\right)
\end{equation}





\end{document}
